\documentclass{book}

%% Language and font encodings
\usepackage[english]{babel}
\usepackage[utf8x]{inputenc}
\usepackage[T1]{fontenc}
\usepackage{fancyhdr}


%% Sets page size and margins
\usepackage[a4paper,top=2cm, bottom=2cm,left=3cm,right=3cm,marginparwidth=1.95cm]{geometry}
\usepackage{xcolor}

%% Useful packages
\usepackage{amsmath}
\usepackage{graphicx}
\usepackage[colorinlistoftodos]{todonotes}
\usepackage[colorlinks=true, allcolors=blue]{hyperref}
\usepackage{amssymb}
\usepackage{color}
\usepackage{setspace}



\setcounter{chapter}{5}
\setcounter{section}{3}
\newcounter{problem1}
\setcounter{problem1}{13}
\newcommand{\problem}{\par\addtocounter{problem1}{1}%
\textbf{Problem \arabic{chapter}.\arabic{problem1} }\quad}
\setcounter{equation}{25}
\setcounter{subsection}{3}

\begin{document}

  \large\textit{5.3 Fractional changes with general exponents} \hfill \textbf{89} \\

\subsection {\Large Limits of validity}

\noindent \Large {The linear fractional-change approximation}
\begin{equation}
\frac{\Delta (x^n)}{x^n}\approx{n}\times{\frac{\Delta x}{x}}
\end{equation}
has been useful. But when is it valid? To investigate without drowning
in notation, write $z$ for $\Delta x$; then choose $x = 1$ to make $z$ the absolute
and the fractional change. The right side becomes $nz$, and the linear
fractional-change approximation is equivalent to
\begin{equation}
(1+z)^n\approx{1+nz}.
\end{equation}
\Large The approximation becomes inaccurate when z is too large: for example,
when evaluating $\sqrt{1+z}$ with $z = 1$ (Problem 5.22). Is the exponent n
also restricted? The preceding examples illustrated only moderate-sized
exponents: $n = 2$ for energy consumption (Section 5.2.3), −2 for fuel
efficiency (Problem 5.18), −1 for reciprocals (Section 5.3.1), 1/2 for square
roots (Section 5.3.2), and −2 and 1/4 for the seasons (Section 5.3.3). We
need further data.\par\medskip
 $\blacktriangleright$ \textit{What happens in the extreme case of large exponents?}\par\medskip
\Large With a large exponent such as $n = 100$ and, say, $z = 0.001$, the approximation
predicts that $1.001100\approx 1.1$---close to the true value of 1.105 . . .
However, choosing the same n alongside $z = 0.1$ (larger than 0.001 but
still small) produces the terrible prediction\\[12pt]
\begin{equation}
\underbrace{1.1^{100}}_{(1+z)^n} = 1+\underbrace{100\times0.1}_{nz} = 11;
\end{equation}
\Large $1.1^{100}$ is roughly 14,000, more than 1000 times larger than the prediction.
Both predictions used large $n$ and small $z$, yet only one prediction was
accurate; thus, the problem cannot lie in $n$ or $z$ alone. Perhaps the culprit
is the dimensionless product $nz$. To test that idea, hold $nz$ constant while
trying large values of $n$. For $nz$, a sensible constant is 1---the simplest
dimensionless number. Here are several examples.
\begin{equation}
\begin{split}
1.1^{10}\approx 2.59374 &,\\
1.01^{100}\approx 2.70481 &,\\
1.001^{1000}\approx 2.71692 &.
\end{split}
\end{equation}

%--------------------------------------------------------------------------------------------------------------------------------

\newpage


\large\textbf{90} \hfill \textit{5 Taking out the big part} \\

\Large\noindent In each example, the approximation incorrectly predicts that $(1 + z)^n = 2$.\par\medskip
$\blacktriangleright$ \textit{What is the cause of the error?}\par\medskip
\begin{minipage}{\textwidth}
 \Large\noindent  To find the cause, continue the sequence beyond\\
$1.001^{1000}$ and hope that a pattern will emerge: The\\
values seem to approach $e = 2.718281828$ . . .., the\\
base of the natural logarithms. Therefore, take the\\
logarithm of the whole approximation.\\
\begin{equation}
\ln{(1+z)^z}=n\ln{(1+z)}
\end{equation}
Pictorial reasoning showed that $\ln{(1 + z)}\approx {z}$ when\\
$z \ll 1$ (Section 4.3). Thus, $n \ln{(1 + z)}\approx {nz}$, making\\
$(1 + z)n\approx {e^{nz}}$. This improved approximation\\ 
\end{minipage} 
\Large\textrm{explains why the approximation $(1 + z)^n\approx 1 + nz$ failed with large $nz$:
Only when $nz \ll 1$ is $e^{nz}$ approximately $1 + nz$. Therefore, when $z \ll 1$
the two simplest approximation are}
\begin{equation}
(1 + z)^n \approx
\begin{cases}
1+nz, & \text{($z\ll  1$ and $nz\ll 1$),} \\
e^{nz}, & \text{($z\ll 1$ and $nz$ unrestricted).}
\end{cases}
\end{equation}
\begin{minipage}{\textwidth}
\noindent The diagram shows, across the whole\\
$n$--$z$ plane, the simplest approximation\\
in each region. The axes are logarithmic\\
and $n$ and $z$ are assumed positive:\\
The right half plane shows $z\gg 1$, and\\
the upper half plane shows $n\gg 1$. On\\
the lower right, the boundary curve is\\
$n \ln{z} = 1$. Explaining the boundaries\\
and extending the approximations is an\\
instructive exercise (Problem 5.28).\\
\end{minipage}\par\medskip
\colorbox{lightgray}{
\begin{minipage}{\textwidth}
\Large{{\bf \problem  Explaining the approximation plane}}

\large In the right half plane, explain the $n/z = 1$ and $n \ln{z} = 1$ boundaries. For the
whole plane, relax the assumption of positive $n$ and $z$ as far as possible.
\\
\Large{{\bf \problem  Binomial-theorem derivation}}

\large Try the following alternative derivation of $(1+z)^{n}\approx e^{nz}$ (where $n\gg 1$). Expand
$(1 + z)^{n}$ using the binomial theorem, simplify the products in the binomial
coefficients by approximating $n$--$k$ as $n$, and compare the resulting expansion
to the Taylor series for $e^{nz}$.

\end{minipage}}

%--------------------------------------------------------------------------------------------------------------------------------

\newpage

\pagestyle{fancy}

\renewcommand{\headrulewidth}{0pt}
\fancyhf{}

  \large\textit{5.4 Successive approximation: How deep is the well?} \hfill \textbf{91} \\

\section {\Large Successive approximation: How deep is the well?}

\Large The next illustration of taking out the big part emphasizes successive
approximation and is disguised as a physics problem

\begin{quote}
 \normalsize You drop a stone down a well of unknown depth h and hear the splash 4 s
later. Neglecting air resistance, find h to within 5\%. Use $c_s = 340 m s^{-1}$ as
the speed of sound and $g = 10 m s^{-2}$ as the strength of gravity.
\end{quote}

\noindent Approximate and exact solutions give almost the same well depth, but
offer significantly different understandings.

\subsection {\Large Exact depth}
\Large\noindent The depth is determined by the constraint that the $4 s$ wait splits into two
times: the rock falling freely down the well and the sound traveling up
the well. The free-fall time is $\sqrt{2h/g}$ (Problem 1.3), so the total time is
\begin{equation}
T=\underbrace{\sqrt{\frac{2h}{g}}}_{\textrm{rock}}+\underbrace{\frac{h}{c_s}}_{\textrm{style}}
\end{equation}
\Large\noindent To solve for $h$ exactly, either isolate the square root on one side and square
both sides to get a quadratic equation in $h$ (Problem 5.30); or, for a less
error-prone method, rewrite the constraint as a quadratic equation in a
new variable $z =\sqrt{h}$.\par\medskip
\colorbox{lightgray}{
\begin{minipage}{\textwidth}
\Large{{\bf \problem Other quadratic}}
\large Solve for $h$ by isolating the square root on one side and squaring both sides.
What are the advantages and disadvantages of this method in comparison with
the method of rewriting the constraint as a quadratic in $z = \sqrt{h}$?
\end{minipage}}\par\medskip
\Large As a quadratic equation in z = √h, the constraint is\\
\begin{equation}
\frac1{c_s}z^2+\sqrt{\frac2{g}}z-T=0
\end{equation}
\Large Using the quadratic formula and choosing the positive root yields\\
\begin{equation}
h=\frac{-\sqrt{2/{g}}+\sqrt{2/{g}+{4T}/{c_s}}}{2/{c_s}}
\end{equation}
\Large Because $z^2 = h$,

\newpage

\large\textbf{92} \hfill \textit{5 Taking out the big part} \\

\begin{equation}
h= \left( \frac{-\sqrt{2/{g}}+\sqrt{2/{g}+{4T}/{c_s}}}{2/{c_s}}\right)^2
\end{equation}

\Large Substituting $g=10ms^{-2}$ and $c_s=340ms^{-1}$ gives $h\approx71.56m$.\\
Even if the depth is correct, the exact formula for it is a mess. Such high-\\
entropy horrors arise frequently from the quadratic formula; its use often\\
signals the triumph of symbol manipulation over thought. Exact answer,\\
we will find, may be less  useful than approximate answers.

\subsection{\Large Approximate depth}

\Large To find a low-entropy, approximate depth, identify the big part---the\\
most important effect. Here, most of the total time is the rock's free\\
fall: The rock' s maximum speed, even if it fell for the entire $4 s$, is only\\
$gT = 40ms^{-1}$, which is far below $c_s$. Therefore, the most important effect\\
should arise in the extreme case of infinite sound speed.\\[6pt]
 $\blacktriangleright$ \textit{If $c_s = \infty$, how deep is the well?}\\[6pt]
\Large In this zeroth approximation, the free-fall time $t_0$ is the full time $T = 4 s$,
so the well depth $h_0$ becomes\\[6pt]
 $\blacktriangleright$ \textit{Is this approximate depth an overestimate or underestimate? How accurate is it?}\\[6pt]
\begin{equation}
{h_0=\frac1{2}g{t^2}_0=80m}.
\end{equation}
\Large This approximation neglects the sound-travel time, so it overestimates
the free-fall time and therefore the depth. Compared to the true depth
of roughly $71.56 m$, it overestimates the depth by only $11\%$--reasonable
accuracy for a quick method offering physical insight. Furthermore, this
approximation suggests its own refinement.\\[6pt]
 $\blacktriangleright$ \textit{How can this approximation be improved?}\\[6pt]
\begin{minipage}{\textwidth}
\Large To improve it, use the approximate depth $h_0$ to approx-\\
imate the sound-travel time.\\
\begin{equation}
{t_{\textrm{sound}}\approx \frac{h_0}{c_s}\approx 0.24}.
\end{equation}
\end{minipage}
\Large The remaining time is the next approximation to the free-fall time.




\end{document}